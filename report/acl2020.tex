\documentclass[11pt,a4paper]{article}
\usepackage[hyperref]{acl2020}
\usepackage{times}
\usepackage{latexsym}
\renewcommand{\UrlFont}{\ttfamily\small}

\usepackage{microtype}

\aclfinalcopy 
%\def\aclpaperid{***} %  Enter the acl Paper ID here

%\setlength\titlebox{5cm}


\newcommand\BibTeX{B\textsc{ib}\TeX}

\title{Comparing Sentiment Analysis Methods}

\author{Alp Gökçek \\
  Department of Computer Engineering\\
  MEF University\\
  Istanbul, Turkey \\
  \texttt{gokcekal@mef.edu.tr} \\\And
  Erdal Sidal Doğan \\
  Department of Computer Engineering\\
  MEF University\\
  Istanbul, Turkey \\
  \texttt{doganer@mef.edu.tr} \\}
\date{\today}

\begin{document}
\maketitle
\begin{abstract}
abstract
\end{abstract}

\section{Introduction}
Sentiment Analysis is an NLP technique to determine whether the given piece of text is positive, negative or neutral. Prior to World Wide Web and embracement of the social media platforms such as Twitter, Facebook, forums etc. by vast majority, studies on this area were very limited. Web has changed the way people express their opinions about an entity, hence, enabled researchers to collect the opinionated texts from Web and conduct their research.\\

In today's world, Sentiment Analysis has been incorporated by hundreds of companies and has a wide range of use in academia. For companies that want to monitor their reputation and opinions about their products/services, sentiment analysis offer these companies to monitor social media in real time and assess their view in public in real time.\\

It is common to classify sentences into two or three groups, Positive, Neutral, Negative.

As an example, text snippet from a hotel review in Manhattan can be seen below.\\

\parbox{0.45\textwidth}{
	\centering
	\small
	\textit{
		"The king suite was spacious, clean, and well appointed. The reception staff, bellmen, and housekeeping were very helpful. Requests for extras from the maid were always provided.\ldots The neighborhood is the best for shopping, restaurants." \cite{ronen}
	}
}\\

Overall, the review presents a positive opinion about the hotel. A sentiment system accepts such review as an input and assigns a score to the positivity of the sentiment in the text.

\section{Motivation}
Given that sentiment analysis has gained such popularity and being used for various of tasks, there are many individuals and organizations that is implementing such systems for their works. However, initial challenge they encounter is the selection of the solution approach. As in many other engineering problems, each solution method brings its unique trade-offs. \\

We believe it is valuable to provide a comparison between the state-of-the-art solution methods and ease the selection process of people who seek to implement their own customized sentiment analysis systems.

\section{Problem Definition}
problem definition text

\section{Evaluation}
evaluation text

\section{Conclusion}
conclusion text

\bibliography{anthology,acl2020}
\bibliographystyle{acl_natbib}

\end{document}
